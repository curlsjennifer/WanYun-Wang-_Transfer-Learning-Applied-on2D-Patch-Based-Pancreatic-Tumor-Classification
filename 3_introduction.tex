\chapter{Introduction中文}
\section{Background}
\subsection{Technical background}
The medical imaging field has witnessed a new generation of computer-aided systems that show incredible performance with the revival of neural networks owing to the development of parallel computing technique. A Survey on Deep Learning in Medical Image Analysis \cite{litjens2017survey} surveyed the use of deep learning techniques such as image classification, object detection, segmentation, registration. ((add two more thesis)) 

\subsection{Clinical background}
Pancreatic cancer is a frequent cancer but hard to process, so few thesis discussed about it. Pancreatic cancer is the second most frequent cancer of digestive system which caused 45,750 deaths in the United States in 2019\cite{siegel2019cancer}. Pancreatic cancer is hard to process because the pancreas’ shape, size and location in the abdomen can vary seriously among patients. These factors make the model less accurate and robust. \cite{Roth2015-mi} ((how we solve this))

\subsection{Transfer Learning}
After developing a workable pancreatic neural network model on NTUH dataset (source dataset), I use fine-tuning, a kind of transfer learning technique, to create another model on smaller external dataset (target dataset). Due to the efforts of the doctors in NTUH, we have sufficiently large pancreatic dataset.  We want to use this precious data to help train models for other pancreatic dataset. I choose fine-tuning method because of this thesis Learning without Forgetting [LH17]. The authors compare four methods, including joint training, feature extraction, learning without forget-ting and fine-tuning.  The fine-tuning method performs well in external data, and also has excellent testing efficiency.

Another thesis in 2017 applies fine-tuning on ultrasound images and has an obvious enhance on model performance.  [CWB+17] Since  2018,  our  team  continues  collecting  new  external  data  and  labeling  the images.   If  we  can  select  patient  data  that  provides  information  most,  the  model performance increases more for certain amounts of labeled data.  [ZSZ+17] provides AIFT method to do data selection.The thesis applies fine-tuning model on pancreatic images,  also data selection will be applied.

XXX  transfer learning  in medical image \\
The research team of my thesis advisor Dr. Wang started working on pancreatic cancer since 2018 and had developed 2D patch-based tumor classification model based on patient data from National Taiwan University Hospital (NTUH). But when we tested our model using data from other data sources, the performance of the classification model is worse than the previous results. (In this thesis, we mainly use the area under receiver operating characteristic curve (AUC) to evaluate the performance.) Our goal is to improve the performance of model working on external data using fine-tuning technique so that we can apply our model to patient data from different datasets and different countries in the future.

\section{Study objectives}
((4))\\
Consider a source dataset and an target dataset. Source dataset has sufficient data to train a model, and target dataset doesn't have sufficient data to train a workable model. The purpose of the thesis is
\begin{itemize}
    \item Use source data and target data to build a model that has high performance on target test set.
    \item Evaluate how many patients are needed to improve the performance.
    \item The selection method in thesis \cite{zhou2017fine} selects a set of target data that can help more on fine-tuning model. 
    % 
\end{itemize}
