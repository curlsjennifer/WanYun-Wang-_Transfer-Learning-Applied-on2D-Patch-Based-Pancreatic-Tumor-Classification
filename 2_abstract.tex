

%\usepackage[a4paper,left=3cm,top=3cm,right=3cm,bottom=2cm,nohead]{geometry}

\doublespacing

\begin{center}
\Large{{Abstract}}\\
\end{center}
((Structured summary of study design, methods, results, and conclusions))\\
Pancreatic  cancer  is  the  second  most  frequent  cancer  of  the digestive  system, and it's an excellent choice to apply artificial intelligence to help doctors with tumor detection. \\
I'm in Dr. Weichung Wang's research team and work on this detection problem with other master students. But all medical images, including pancreas CT, are precious and hard to get. With the help of doctors in National Taiwan University Hospital (NTUH), we get enough data to  train an accurate convolution neural network. In the future we need to apply our model to different dataset, but the performance of detection model decreases when testing on other datasets. Our goal is to enhance the performance of model on other dataset. \\
In this paper, we seek to answer the following core question in the context of medical image analysis: How much fine-tuning improves the performance on external dataset? How much data we need to get a certain performance? Our experiments consistently demonstrated that, as we get more and more external data, we can get a better prediction result on external data. We can at most enhance the area under the receiver operating characteristic curve (AUC) up to 0.04. This paper can help to evaluate how much external data we need to get a satisfied result and help our team to promote our neural network result worldwide.

