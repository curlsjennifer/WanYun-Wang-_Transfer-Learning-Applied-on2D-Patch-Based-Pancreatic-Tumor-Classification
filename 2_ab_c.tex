
\linespread{1.25}
\begin{CJK}{UTF8}{bkai}

\addcontentsline{toc}{chapter}{摘要}
\begin{center}
\Large{{摘要}}\\
\end{center}
胰腺癌是消化系統中第二常見的癌症,人工智能通常用於協助醫生進行腫瘤檢測。由於包括胰腺在內的所有醫學圖像都是珍貴且難以獲得的,因此我嘗試對圖像進行轉移學習以提高外部數據集的性能。在醫生的幫助下,我獲得了足夠的數據來訓練準確的卷積神經網絡作為基礎分類模型。在此模型的基礎下,我需要將模型使用於不同的數據集,但是在其他數據集上進行測試時,分類模型的性能會降低。我的目標是使用遷移學習方法基於基礎分類模型創造能在外部數據集有精確結果的模型。 \\

在本文中,我們試圖在醫學圖像分析的背景下回答以下核心問題:微調,一種常見的遷移學習方法,如何提高外部數據集的性能;為了獲得一定的性能,我們需要多少數據。我們的實驗一致表明,隨著我們獲得越來越多的外部數據,我們可以獲得對於外部數據更好的預測結果。本文可以幫助評估獲得滿意預測表現所需的外部資料量,並在全球範圍內推廣先前開發的神經網絡模型。此外,還使用增量學習來評估哪個外部掃描圖像(無手動標記)最能改善性能,以便我們可以選擇有價值的數據進行標記。\\

在本文中,我發現混合數據方法和微調方法可以顯著提高外部數據的模型性能。以微調方法為基礎的增量選擇方法可以稍微改善模型性能,但總體而言,它的性能並不比基本的微調更好。\\

關鍵詞:醫療影像、人工智慧、類神經網路、深度學習、遷移學習
\end{CJK}
\doublespacing
