\chapter{Preliminary}

\section{Definition and Notation}
\section*{This is different}
    %==========================
    \subsection*{Examples}
    \label{sec:examples}
    %==========================
    
    \begin{description}
        %\item[Shortest Vector Problem(SVP)] Given a lattice basis $\bm{B}$ $\in{\mathbb{R}^{m\times n}}$, find the shortest nonzero vector in \latticeb{B}.
        %
        \item[Problem] Descriptions
        %
    \end{description}
    
    %=====================
    \paragraph{Problem Property}
    %=====================

\begin{algorithm}
    \label{tab:alg}
    \centering
    \begin{algorithmic}[H]
        \Require{ basis {B}($\vec{b}_1,...,\vec{b}_n$), PrunedBound $R_1^2 \leq R_2^2 \leq ... \leq R_n^2$ }
        \Ensure{The coefficients $(x_1,...,x_n)$ of the basis satisfying the Pruned Bound}
        \State  Compute Gram-Schmidt orthogonalization $\mu$ of basis B  
        \State    $\sigma \leftarrow (0)_{(n+1) \times n}$
        \While{$true$}
            \State $\rho_k = \rho_{k+1} + (v_k - c_k)^2 \dot \| {b_k^*} \|^2$ 
            \If{ $\rho_k \leq R_{n+i-k}^2$}
                \If{ $k = 1$} 
                    \State return $(v_1,...,v_n)$ 
                \Else  
                    \State $k \leftarrow k - 1$
                    \State $r_{k-1} \leftarrow max(r_{k-1}, r_{k})$
                    \For{$i = r_k$ downto $k+1$}
                        \State $\sigma_{i,k} \leftarrow \sigma_{i+1,k} + v_k \mu_{i,k}$ 
                    \EndFor
                    \State $c_k \leftarrow -\sigma_{k+1,k}$ 
                    \State $v_k \leftarrow \lfloor c_k \rceil$; $w_k = 1$;
                \EndIf
            \Else
                \State $k \leftarrow k+1 $
            \EndIf 
        \EndWhile
    \end{algorithmic}
    \vspace*{1em}
    \caption{This is an example of algorithm.}
\end{algorithm}



This is an example of equations.

\begin{equation*}
\| {\sum_{t=1}^{n} u_t \vec{b}_t} \| = \min_{x\in \mathbb{Z}^{n}}  \| \sum_{t=1}^{n} x_t \vec{b}_t \|
\end{equation*}
we replace all $\vec{b}_{t}$ by their orthogonalization, i.e., $\vec{b}_{t} = \vec{b}_{t}^{*} + \sum_{j=1}^{t-1} \mu_{t,j} \vec{b}_{j}^{*}$ and get a degree.










